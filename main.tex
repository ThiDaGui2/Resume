\documentclass[11pt, a4paper]{moderncv}
\usepackage{tikz}
\usepackage[utf8]{inputenc}
  
\author{damien.didier}
\date{February 2022}

%\usepackage{fontspec}
\usepackage[inline]{enumitem}
\usepackage[left=1.5cm,right=1.5cm,top=1.5cm,bottom=1.5cm]{geometry}
\moderncvstyle{classic}
\moderncvcolor{blue}

\firstname{Damien}
\familyname{\textsc{Didier}}

\email{damien.didier@epita.fr}
\phone[mobile]{+33 (0)7 70 42 61 54}
%\phone[fixed]{}
%\social[github]{}
%\social[linkedin]{}

\address{23 sentier de la Commune}{94800 \textsc{Villejuif}}{\textsc{France}}
\photo[64pt]{damien.didier.jpg}


\newcommand\topVsp{-40pt}%{0pt}
\newcommand\secVsp{-5pt}%{0pt}
\newcommand\secpostVsp{-5pt}%{0pt}

\newcommand{\cvtag}[1]{%
  \tikz[baseline]\node[anchor=base,draw=black,rounded corners,inner xsep=1ex,inner ysep=0.75ex,text height=1.5ex,text depth=.25ex]{#1};
  \vspace{0.25ex}
}

\begin{document}

\makecvtitle

\vspace{\topVsp}
\section{Formation} 
\vspace{\secpostVsp}
\cventry{Février 2019 - Aujourd'hui}{Ecole d'Ingénieur}{EPITA}{\textsc{Le kremlin-bicêtre}}{}{}
\cventry{Septembre 2017 - Juin 2018}{Classe Préparatoire PTSI}{Lycée Jean-Baptiste Say}{\textsc{Paris}}{}{}
\cventry{Juin 2017}{Diplôme du Baccalauréat général série S}{Lycée Saint Thomas de Villeneuve}{\textsc{Chaville}}{mention \emph{Très Bien}}{dont 20 en Informatique et Science du numérique (ISN)}


\vspace{\secVsp}
\section{Expériences professionnelles} % Résultats
\vspace{\secpostVsp}
\cventry{Octobre 2020 - Novembre 2020}{Stage ouvrier}{Total, Data Lab}{\textsc{Tour Coupole, la Défense}}{Aide à la labellisation de données dans le but d'entrainer des reseaux de neurones}{}

\vspace{\secVsp}
\section{Savoirs-faire}
\vspace{\secpostVsp}
%\subsection{Computing}
%\cvitem{Langages de Programmation}{
%    \begin{itemize*}
%    \item À l'aise :
%    \cvtag{C}
%    \cvtag{Python}
%    \item Mais aussi : \begin{itemize*}
%    \item C\#
%    \item Rust
%    \item Caml
%    \end{itemize*}            
%    \end{itemize*}
%    }
%\cvitem{Modélisation}{
%    \begin{itemize*}
%    \item Blender
%    \end{itemize*}
%    }
%\vspace{\secVsp}
\cvitem{Langages de Programmation}{
    À l'aise :
    \cvtag{C}
    \cvtag{Python}
    \cvtag{PostgreSQL}
    \cvtag{Java}
    Mais aussi :
    \cvtag{C\#}
    \cvtag{Rust}
    \cvtag{Caml}}
\cvitem{Bureautique}{
    \cvtag{Git}
    \cvtag{LateX}
    \cvtag{Linux}
    \cvtag{Microsoft Office}}
\cvitem{Création Numérique}{
    \cvtag{Blender}
    \cvtag{Gimp}}
    
\section{Langues}
\vspace{\secpostVsp}
\cvlanguage{Anglais}{Courant}{Niveau B2 certifié en 2015 \emph{(Cambridge English: First)}}
\cvlanguage{Espagnol}{Basique}{}
\vspace{\secVsp}


\section{Extra}
\subsection{Passions}{
    \cvitem{Informatique}{
    \begin{itemize*}
    \item Rendu 3D en temps réel
    \item Modélisation 3D
    \item Jeux vidéos
    \end{itemize*}
    }
    \cvitem{Travaux manuels}{
    \begin{itemize*}
    \item Origami
    \item Modélisme
    \item Sculpture
    \item Dessin
    \end{itemize*}
    }

}
\subsection{Scoutisme (12 ans)}
\cvitem{2018}{Chantier en Moldavie, association \emph{Vent d'Est} (pose de carrelage dans une future fabrique de confiture)}
\cvitem{2017}{Chantier \emph{île de Cezon}, fort de type Vauban (déblayage d'une caserne du XIX siècle)}


\subsection{Musique (10 ans)}
\cvitem{Conservatoire}{
    \begin{itemize*}
    \item Trompette
    \item Solfège
    \end{itemize*}}

\subsection{Projets}
%\cvitem{Projet ISN}{Jeu d'exploration de donjon en tour par tour avec des éléments de jeu de rôle, Python}
%\cvitem{OCR}{Implementation d'un reseau de neurone pour faire de la reconnaissance de caractère, C}
%\cvitem{Malloc}{Implementation des fonctions d'allocation dynamique de mémoire du langage C, C}
\cvitem{R.A.G.E.}{Application de visualisation d'objet 3D en temps réel en console, C}
\cvitem{42sh}{shell POSIX, C}
%\cvitem{Blender}{Modélisation d'une arme de \emph{Warhammer 40,000} (\emph{Bolter})}
\subsection{Lectures}
\cvlistitem{\emph{Real-time Rendering}, Tomas AKENINE-MÖLLER, \footnotesize{fourth edition}}
\cvlistitem{\emph{The art of 3-D computer animation and imaging}, Isaac Victor KERLOW, \footnotesize{second edition}}
\cvlistitem{\emph{Game Engine Architecture}, Jason GREGORY, \footnotesize{third edition}}

\end{document}
